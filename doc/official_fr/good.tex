\section{Points positifs}
\begin{enumerate}
	\item Le~moteur du jeu est profondément modulaire :
	\begin{enumerate}
		\item Il est facile d'ajouter de nouveaux succès et des ensembles complexes 
				de conditions pour chacun d'entre eux : les~succès ne sont pas 
				directement liés aux~jeux mais plutôt déclenchés à travers des événements : 
				un~mouvement, une~victoire, une~défaite, un~retour en arrière, etc.
		\item Des jeux supplémentaires mais semblables aux jeux déjà implémentés peuvent facilement 
				être ajoutés : il suffit de créer une~nouvelle classe pour chacun d'eux ainsi que 
				pour chacun de leurs mouvements spécifiques.
		\item Les~IA sont interchangeables : à partir du moment où un~jeu propose 
				une~liste de mouvements légaux et évalue la~situation courante, 
				n'importe quelle IA est compatible avec lui.
	\end{enumerate}
	
	\item Les~textures ont entièrement été faites par nos~soins.

	\item Tout le~projet est capable de tenir dans un~fichier \texttt{jar}, ce qui le rend plus facile à partager.

	\item Le~système de retour en arrière a été implémenté dès le début et fonctionne partout.

	\item Le~débogage des IA ainsi que la triche sont facilités par la~présence d'un~mode de fonctionnement et de raccourcis dédiés.

	\item Le~déroulement des jeux est totalement isolé de l'interface graphique : cette dernière 
			ne fait qu'instancier des objets génériques (chose possible grâce au polymorphisme) et 
			utiliser les~méthodes qu'ils présentent. 

	\item Le~code est bien organisé grâce à l'utilisation maximale que nous avons faite de la~POO et abondamment documenté en utilisant Javadoc.
	
	\item Les~statistiques des intelligences artificielles sont exportables dans gnuplot.
\end{enumerate}
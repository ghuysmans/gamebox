\section{Apports personnels}

\subsection{Mathieu Leclercq}
Pour ma part, j'ai apprécié le~fait de devoir travailler en groupe et de concevoir un~projet dans son~entièreté,
devoir penser aux moyens les plus efficaces pour que le~programme respecte le~cahier des charges. 
Ce~projet m'a aussi permis d'utiliser certaines notions de Java vues au cours et que l'on n'avait pas encore eu 
l'occasion d'utiliser durant les séances de travaux pratiques.   

\subsection{Guillaume Huysmans}
Ce~projet n'était pas pour moi le~premier mais il m'a apporté un~éclairage nouveau sur le~développement logiciel : 
tout d'abord, je n'avais jamais développé ni en un~langage orienté objets, ni avec un autre développeur. 
Cela implique une toute autre manière de travailler : il faut obligatoirement s'interroger en profondeur (et se mettre d'accord) 
sur la~structure du projet avant de commencer à coder afin d'éviter tout conflit (par exemple, un~changement radical 
de fonctionnement qui impliquerait la~réécriture de toute une~partie du~code). 
L'UML produit lors de cette~phase d'analyse m'a été d'une~grande aide pour me tenir à ce qui avait été prévu. 
Le~code que nous écrivons doit en outre être suffisamment documenté pour éviter toute perte de temps.
Heureusement, la~POO rend bien plus facile le~développement collaboratif à travers l'encapsulation : ce qui se passe à l'intérieur 
d'un~objet n'a pas d'influence sur le~reste du code, ce qui permet d'effectuer des modifications sans tout <<~casser~>>. 
Ce jeu nous a également donné l'occasion d'apprendre à utiliser efficacement 
un~VCS\footnote{\emph{Version Control System}, système de contrôle de versions} 
afin de coordonner notre~travail le plus efficacement possible : la~synchronisation manuelle des codes sources par e-mail ou 
clé USB est un~travail lourd, difficile, source d'erreurs et d'énormes frustrations.
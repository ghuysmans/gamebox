\section{Points négatifs}
\begin{itemize}

	\item Malgré l'utilisation de l'algorithme Negamax, l'IA (même à son~niveau maximum) reste faible : 
            l'évaluation trop manichéenne en est sans aucun doute une des~causes. En effet, Negamax est fait pour 
            détecter de subtiles différences entre des situations de jeu. Ici, il n'y en a que trop peu ! 
            La~seule chose qui distingue deux situation c'est leur~profondeur : pour <<~éviter la~casse~>>, 
            le~score est systématiquement divisé par $2^d$, $d$ étant le~niveau de récursion où il a été calculé. 
            
    \item Nous avons mal géré notre~temps : l'erreur dans la conception de l'heuristique (évaluation d'une situation de jeu) 
            aurait dû être trouvé plus tôt, nous aurions dû passer un peu moins de temps à l'analyse de détails 
            et plus de temps à tester nos~algorithmes... Ce qui semblait plus simple à implémenter 
            se révèle en réalité <<~simplet~>>.
            
    \item Le~clic n'est pas pris lorsque la~souris se déplace...
    
\end{itemize}
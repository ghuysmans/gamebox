\section{Choix argumentés}

\subsection{Organisation interne}
Le même objet est utilisé pour représenter le~plateau de tous les~jeux. Ce~plateau n'est jamais manipulé directement par ces~derniers :
les~objets représentant les~jeux génèrent des listes de coups possibles qui seront ensuite utilisés par les différentes intelligences artificielles.
Pour savoir si un~mouvement est autorisé, il nous suffit de vérifier qu'il appartient à cette~liste. 

Du fait que les~jeux proposent tous une~liste de coups autorisés par les~règles, les~intelligences artificielles sont indépendantes des jeux :
il suffit de modéliser les~différents types de mouvements et leurs~conséquences pour permettre \emph{à la fois} à un~joueur humain et au logiciel de jouer !

Le~jeu de \ttt{} et \cf{} ont une~caractéristique commune : les~conditions de victoire sont identiques, seul le~nombre de jetons diffère !
\cf{} ne fait qu'ajouter une~contrainte lorsqu'un~jeton est joué, c'est pourquoi il hérite de \ttt.

L'historique est géré sous la~forme d'une~pile : on ne peut défaire un~mouvement que lorsque c'est le~dernier effectué. 
Les~intelligences artificielles s'en servent abondamment pour explorer les différentes possibilités. 
Cela implique que l'affichage ainsi que les événements ne doivent pas toujours être mis à jour : 
lorsqu'une~IA <<~réfléchit~>>, les~mouvements effectués ne le sont pas réellement (ils sont \emph{virtuels})...

Les~succès et les~différents messages (sonores ou textuels) sont gérés à travers un~seul mécanisme : 
les~événements. Ces~derniers sont constitués de plusieurs compteurs nommés, mis à jour par le~jeu lui-même. 

L'utilisation du design patern \texttt{Observer} nous évite des incohérences entre l'interface graphique et l'état interne des jeux.

La~sérialisation nous a permis de facilement sauvegarder et enregistrer les~parties en cours et les~profils des joueurs. 
Les~classes composant le~design pattern \texttt{Observer} ont cependant dû être réécrites puisqu'\texttt{Observable} n'est pas sérialisable, 
ce qui nous empêchait de sauvegarder les~liens entre les différentes conditions et le~jeu en cours d'exécution.


\subsection{Sauvegarde et chargement}
Le fait qu'un objet \texttt{JPanel} puisse être sérialisé peut apparaître comme une solution miracle à ce problème 
mais nous avons pourtant pris la~décision de ne pas faire reposer notre mécanisme de sauvegarde là-dessus.

Premièrement, l'interface graphique doit pouvoir évoluer sans pour autant interférer avec les~sauvegardes effectuées auparavant : 
qui voudrait mettre à jour son~jeu et voir disparaître ses~profils de joueurs ?

Ensuite, nous sommes capables de recréer tous les~composants de l'interface graphique nécessaires à l'utilisation d'un jeu 
à partir d'une instance sauvegardée d'un objet \texttt{Game} : une énumération associant la~classe de chaque jeu jouable 
aux informations nécessaires à la~création du~\texttt{BoardPanel} correspondant est chargée dans le~constructeur de \texttt{Main}, 
il nous suffit de trouver parmi les~éléments celui qui correspond à la~classe de l'objet sauvegardé ! C'est également cette même structure 
%FIXME orthographe ? <<~sélection de jeux~>> ??
qui nous permet de construire le menu de sélection de jeu entièrement à la volée.


\subsection{Intelligences artificielles}
Comme il nous était demandé d'implémenter plusieurs <<~intelligences artificielles~>>, 
nous en avons choisi trois, classées ici par niveau selon l'ordre croissant :

\begin{enumerate}
	\item \emph{Hasard} : elle sélectionne un~coup au~hasard parmi les~coups légaux.
    \item \emph{Premier} : elle sélectionne le premier coup légal de la~liste. Le~balayage est fait,
    \begin{itemize}
        \item pour \oth{} et \ttt{} : de gauche à droite et de haut en bas.
        \item pour \cf{} : de gauche à droite.
    \end{itemize}
    \item \emph{Negamax} : elle sélectionnera le meilleur coup possible en simulant les~coups de son~adversaire sur plusieurs niveaux de récursivité.
\end{enumerate}
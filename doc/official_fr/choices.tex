\section{Choix argumentés}
Le même objet est utilisé pour représenter le~plateau de tous les~jeux. Ce~plateau n'est jamais manipulé directement par ces~derniers :
les~objets représentant les~jeux génèrent des listes de coups possibles qui seront ensuite utilisés par les différentes intelligences artificielles.
Pour savoir si un~mouvement est autorisé, il nous suffit de vérifier qu'il appartient à cette~liste. 

Du fait que les~jeux proposent tous une~liste de coups autorisés par les~règles, les~intelligences artificielles sont indépendantes des jeux :
il suffit de modéliser les~différents types de mouvements et leurs~conséquences pour permettre à un~joueur humain ou au logiciel de jouer !

Le~jeu de \textit{tic-tac-toe} et \textit{puissance 4} ont une~caractéristique commune : les~conditions de victoire sont identiques, seul le~nombre de jetons diffère !
\textit{Puissance 4} ne fait qu'ajouter une~contrainte lorsqu'un~jeton est joué, c'est pourquoi il hérite de \textit{tic-tac-toe}.

L'historique est géré sous la~forme d'une~pile : on ne peut défaire un~mouvement que lorsque c'est le~dernier effectué. 
Les~intelligences artificielles s'en servent abondamment pour explorer les différentes possibilités. 
Cela implique que l'affichage ainsi que les événements ne doivent pas toujours être mis à jour : 
lorsqu'une~IA <<~réfléchit~>>, les~mouvements effectués ne le sont pas réellement (ils sont \emph{virtuels})...

Les~succès et les~différents messages (sonores ou textuels) sont gérés à travers un~seul mécanisme : 
les~événements. Ces~derniers sont constitués de plusieurs compteurs nommés, mis à jour par le~jeu lui-même. 

L'utilisation du design patern \texttt{Observer} nous évite des incohérence entre l'interface graphique et l'état interne des jeux.

La~sérialisation nous a permis de facilement sauvegarder et enregistrer les~parties en cours et les~profils des joueurs. 
Les~classes composant le~design pattern \texttt{Observer} ont cependant dû être réécrites puisqu'\texttt{Observable} n'est pas sérialisable.
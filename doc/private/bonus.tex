\section{Fonctionnalités supplémentaires}
Le~diagramme du projet pourra contenir quelques classes bonus mais elle ne devront être effectivement implémentées que si on a terminé le~reste ! Il ne faut pas donner l'impression (même si elle est fausse) qu'on s'est concentrés sur les~détails ! Au besoin, créer des mocks\footnote{Classes vides avec juste ce qu'il faut pour que ça compile et fasse fonctionner le~reste.}. Les~idées sont citées par ordre de priorité.
\subsection{Idées du prof}
\begin{enumerate}
    \item Possibilité de modifier la~taille du plateau.
    \item Possibilité de modifier le~nombre de jetons à aligner pour gagner.
    \item Mode de triche (affichage de <<~conseils~>>, aussi un \textit{debug mode}) activable via un raccourci clavier discret (\textit{Konami Code, just for fun!}).
    \item Effets visuels (animations, images au lieu de formes).
\end{enumerate}
\subsection{Idées originales}
\begin{enumerate}
    \item Après la~remise du projet, mettre ça en commun avec les~autres.
    \item Stockage de l'historique des coups et possibilité de retour en arrière.
    \item Keypad comme interface pour Tic-tac-toe.
    \item Système de succès (trophées, \textit{achievements}...) via des règles.
        \begin{enumerate}
            \item Bruitages basés là-dessus
            \item Commentaires audio/textuels désobligeants
        \end{enumerate}
    \item Réseau
        \begin{enumerate}
            \item protocole TCP/IP (réseau local d'abord) par paquets sérialisés.
                \begin{itemize}
                    \item tchat textuel (\textit{debug mode}, aussi)
                    \item visualisation du profil de l'autre
                    \item partie normale : coups et chrono
                    \item mode spectateur
                \end{itemize}
            \item matchmaking via broadcasting local
            \item classement ELO via un~serveur (POO en PHP, alors !)
        \end{enumerate}
	\item Multilinguisme de l'interface ? Gros morceau, malheureusement.
\end{enumerate}

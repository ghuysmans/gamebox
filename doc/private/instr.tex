\section{Consignes enrichies}
\begin{itemize}
    \item Se baser sur les~règles d'Othello, de Puissance 4 et de Tic-tac-toe données sur leur~page Wikipédia datant du 3~février 2014 à 12h. Comme le disait notre cher prof, <<~Si vous vouliez changer les~règles, fallait le faire [avant]~>>.
	\item Travailler efficacement\footnote{Par exemple, en faisant souvent \texttt{svn update} pour ne pas réinventer l'eau tiède...} à~deux (c'est évalué) et bien gérer\footnote{Planning à définir via Google Calendar (par exemple) !} notre~temps (4~mois, pas plus). La~remise du travail (application documentée) sera faite avant le~16~mai 2014 à~20h (idéalement deux ou trois~jours avant).
	\item Concevoir un~design modulaire via UML : une réflexion profonde à propos des objets sera nécessaire afin de pouvoir faire du <<~recyclage de masse~>>.
    \item Pas de copier-coller, ça doit être lisible (enfin c'est du Java, donc bon...).
	\item Optimiser au maximum les~algorithmes... Non, on n'a pas dormi en algo 1.
	\item Écrire une~documentation correcte (anglais à la~base et traduction en français ?).
        \begin{itemize}
            \item Javadoc : \texttt{@author}, \texttt{@param}, \texttt{@return}, \texttt{@throws}.
            \item Le~rapport en est une~forme (allégée, normalement) et doit être écrit au fur et à mesure.
        \end{itemize}
	\item Coder des~classes de tests pour à peu près tout (par sécurité).
        \begin{itemize}
            \item Obligatoire : partie terminée.
            \item Obligatoire : conditions de victoire.
        \end{itemize}
	\item Diviser l'application en packages, par thème :
        \begin{itemize}
            \item UI (\textit{user interface}) : tout ce qui est graphique.
            \item AI (\textit{artificial intelligence}). Il faudrait que ça soit relativement indépendant.
            \item Moteur (\textit{game engine}) : gestion des différents jeux, coeur du logiciel.
        \end{itemize}
    \item Tenir un~journal avec les~décisions en plus de SVN.
    \item Convention : les~noms des interfaces commencent par un~I majuscule.
    \item Tester sur une~machine de la~salle Escher bien avant la~remise.
        \begin{itemize}
            \item C'est Java 1.6, pas question d'utiliser des~nouveautés !
            \item Tout compiler depuis le~début, bien sûr...
        \end{itemize}
    \item Citer toutes les~références : textes, musique, algorithmes, etc.
\end{itemize}
